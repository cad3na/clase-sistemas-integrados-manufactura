%-------------------------------------------------------------------------------
%	PAQUETES Y OTRAS CONFIGURACIONES
%-------------------------------------------------------------------------------

\documentclass[addpoints, 11pt, letterpaper]{exam} % Tamaño de papel y letra para el documento

\usepackage[utf8]{inputenc} % Los caracteres acentuados se pueden escribir normalmente en el código
\usepackage[T1]{fontenc} % Configuración de fuente de salida
\usepackage{fourier} % Se usa una fuente diferente al default
\usepackage[spanish,es-noquoting]{babel} % Se configura como documento en español
\usepackage{amsmath,amsfonts,amsthm} % Paquetes para escribir formulas matemáticas
\usepackage{wasysym}
\usepackage{graphicx} % Paquetes para incluir imágenes
\usepackage{multirow} % Paquete para unir varias filas en tablas

\usepackage{circuitikz}
\usepackage{tikz}
\usetikzlibrary{arrows}
\usepackage{tkz-euclide}

\usepackage{sectsty} % Paquete para configuración de secciones
\allsectionsfont{\centering \normalfont \scshape} % Los títulos de las secciones son centrados, con la misma fuente y pequeñas mayúsculas

\pagestyle{foot}
\extraheadheight{-1cm}

\setlength\parindent{0pt} % Quita la indentación de los párrafos

\cfoot{}
\firstpagefooter{}{}{}
\runningfootrule

\extrawidth{-1cm}
\marginpointname{ \points}

\pointpoints{ punto}{ puntos}
\pointsinrightmargin
\usehorizontalhalf

\vqword{Pregunta}
\vpgword{Página}
\vpword{Puntos}
\vsword{Obtenidos}
