%----------------------------------------------------------------------------------------
%	PAQUETES Y OTRAS CONFIGURACIONES
%----------------------------------------------------------------------------------------

%----------------------------------------------------------------------------------------
%	PAQUETES Y OTRAS CONFIGURACIONES
%----------------------------------------------------------------------------------------

\documentclass[paper=letter, fontsize=11pt]{scrartcl} % Tamaño de papel y letra para el documento

\usepackage[utf8]{inputenc} % Los caracteres acentuados se pueden escribir normalmente en el código
\usepackage[T1]{fontenc} % Configuración de fuente de salida
\usepackage{fourier} % Se usa una fuente diferente al default
\usepackage[spanish,es-noquoting]{babel} % Se configura como documento en español
\usepackage{amsmath,amsfonts,amsthm} % Paquetes para escribir formulas matemáticas
\usepackage{graphicx} % Paquetes para incluir imágenes

\usepackage{circuitikz}
\usepackage{tikz}
\usetikzlibrary{arrows}

\usepackage{sectsty} % Paquete para configuración de secciones
\allsectionsfont{\centering \normalfont \scshape} % Los títulos de las secciones son centrados, con la misma fuente y pequeñas mayúsculas

\usepackage{todonotes}
\usepackage{microtype}

\usepackage{fancyhdr} % Paquete para personalizar pies y cabeceras de página
\pagestyle{fancyplain} % Todas las páginas con las mismas cabeceras y pies de página
\fancyhead{} % Sin cabecera
\fancyfoot[L]{} % Vacío en la izquierda del pie de página
\fancyfoot[C]{} % Vacío en el centro del pie de página
\fancyfoot[R]{\thepage} % Número de página en el pie de pagina
\renewcommand{\headrulewidth}{0pt} % Sin lineas en la cabecera
\renewcommand{\footrulewidth}{0pt} % Sin lineas en el pie de página
\setlength{\headheight}{13.6pt} % Altura de cabecera

\numberwithin{equation}{section} % Numera ecuaciones en cada sección
\numberwithin{figure}{section} % Numera figuras en cada sección
\numberwithin{table}{section} % Numera tablas en cada sección

\setlength\parindent{0pt} % Quita la indentación de los párrafos

\newcommand{\horrule}[1]{\rule{\linewidth}{#1}} % Comando personalizado para hacer linea horizontal


%----------------------------------------------------------------------------------------
%	TITULO
%----------------------------------------------------------------------------------------

\title{
	\normalfont \normalsize
	\begin{figure}[h]
		\begin{center}
			\includegraphics[width=0.3\textwidth]{./UNITEC.png}
		\end{center}
	\end{figure}
	\textsc{Sistemas Integrados de Manufactura} \\ [25pt]
	\horrule{0.5pt} \\[0.4cm] % Linea horizontal delgada
	\huge Práctica 0 - Ejemplo de Reporte de Práctica \\ % Titulo de la práctica
	\horrule{2pt} \\[0.5cm] % Linea horizontal mas gruesa
}

\author{Roberto Cadena Vega} % Nombre del profesor

\date{ } % Fecha de la práctica

%----------------------------------------------------------------------------------------
%	EMPIEZA EL DOCUMENTO
%----------------------------------------------------------------------------------------

\begin{document}

\maketitle % Imprime el título

%----------------------------------------------------------------------------------------
%	OBJETIVOS
%----------------------------------------------------------------------------------------

\section{Objetivos}

	Familiarizarse con el formato general para la entrega de reporte de prácticas.

%----------------------------------------------------------------------------------------
%	CONOCIMIENTOS PREVIOS
%----------------------------------------------------------------------------------------

\section{Introducción}

	La introducción de un reporte debe de estar compuesta por los conocimientos previos a la ejecución de la práctica. En estas practicas los conocimientos previos estan relacionados con la selección de materiales, procesos de manufactura, etc. y no tanto con la elaboración del diseño y/o simulaciones.

%----------------------------------------------------------------------------------------
%	DESARROLLO
%----------------------------------------------------------------------------------------

\section{Desarrollo}

	El desarrollo es una descripción con tus propias palabras de el procedimiento de construcción del modelo tridimensional y/o simulaciones por medio del software del laboratorio. Se pueden utilizar capturas de pantalla como medio de apoyo gráfico para el desarrollo de las prácticas\footnote{Si se sorprende al alumno utilizando recursos de otro autor, sin dar el correspondiente credito se le serán deducidos puntos.}

    En el caso de las prácticas a realizar en equipo, deben de estar completamente enumerados los integrantes del equipo en cada uno de los reportes de prácticas de sus integrantes.

%----------------------------------------------------------------------------------------
%	CONCLUSIONES
%----------------------------------------------------------------------------------------

\section{Conclusiones}
	Las conclusiones de una práctica deben de ser tanto vivenciales, como orientadas al objetivo marcado al principio de la práctica. Pueden ser tanto positívos, como negatívos; dependiendo de los resultados obtenidos en el laboratorio.

	Recuerda que la calificación de la práctica depende tanto del trabajo que realices en el laboratorio, como de lo preciso de tu reporte.

	La entrega de los reportes se hará en formato PDF (sin excepciones) para los reportes de práctica y los planos de especificación de pieza, y en formato nativo al software (Siemens NX utiliza archivos .prt al igual que Inventor), o bien en un fichero universal como STP.

	La entrega de los reportes se hará por medio del correo:

	\begin{center}
		\textbf{roberto.cadena@my.unitec.edu.mx}. \\
	\end{center}

\begin{center}
	\huge \textthing
\end{center}

%----------------------------------------------------------------------------------------
%	FIN DEL DOCUMENTO
%----------------------------------------------------------------------------------------

\end{document}
