%-------------------------------------------------------------------------------
%	PAQUETES Y OTRAS CONFIGURACIONES
%-------------------------------------------------------------------------------

\input{../../style/practica.tex}

%-------------------------------------------------------------------------------
%	TITULO
%-------------------------------------------------------------------------------

\title{ Practica 6 - Manufactura orientada a ensambles mecánicos }

\author{Roberto Cadena Vega} % Nombre del profesor

\date{ } % Fecha de la práctica

%-------------------------------------------------------------------------------
%	EMPIEZA EL DOCUMENTO
%-------------------------------------------------------------------------------

\begin{document}

\maketitle % Imprime el título
\begin{marginfigure}
	\includegraphics[width=0.8\textwidth]{../../images/UNITEC.png}
\end{marginfigure}

%-------------------------------------------------------------------------------
%	INTRODUCCION
%-------------------------------------------------------------------------------

\section{Introducción}

	\begin{enumerate}
		\item El alumno debe de conocer los conceptos fundamentales alrededor de la manufactura de piezas mecánicas (maquinas herramienta, herramental, etc.).
		\item El alumno debe de conocer las tolerancias relacionadas a ensambles mecánicos.
		\item El alumno debe de conocer las estratégias básicas para la manufactura de piezas mecánicas.
	\end{enumerate}

%-------------------------------------------------------------------------------
%	OBJETIVO
%-------------------------------------------------------------------------------

\section{Objetivo}

	El alumno se entenderá y aplicará las estrategias de manufactura básicas para la fabricación de piezas mecánicas, así como los conocimientos relacionados con su planeación.

%-------------------------------------------------------------------------------
%	DESARROLLLO
%-------------------------------------------------------------------------------

\section{Desarrollo}

	El alumno utilizará la pieza diseñada en la práctica 4 como base para el diseño de estrategias de manufactura que permitan su fabricación por medio de maquinaria CNC por métodos sustractivos. Deberá tomar en cuenta que las piezas a ensamblar son impresas por la tecnoligía FDM y sus tolerancias respectivas.

	El alumno seleccionará el material a utilizar en la planeación de la manufactura, así como las dimensiones del mismo, tomando en cuenta la disponibilidad del material en el mercado, así como la función que debe desarrollar la pieza final.

	El alumno entregará, junto con su reporte de práctica, el archivo .PRT generado en la planeación de la manufactura, así como el código G generado.

%-------------------------------------------------------------------------------
%	CONCLUSIONES
%-------------------------------------------------------------------------------

\section{Conclusiones}

	El alumno deberá dar sus comentarios finales acerca de los conocimientos obtenidos en el desarrollo de esta práctica.

%-------------------------------------------------------------------------------
%	CONCLUSIONES
%-------------------------------------------------------------------------------

\section{Criterios de evaluación}

	El alumno enviará sus reportes de práctica y archivos adjuntos al correo:

	\begin{center}
		\textbf{roberto.cadena@my.unitec.edu.mx} \\
	\end{center}

	El profesor evaluará el reporte práctica, tomando en cuenta la existencia e integridad de los archivos generados en la práctica.

%-------------------------------------------------------------------------------
%	FIN DEL DOCUMENTO
%-------------------------------------------------------------------------------

\end{document}
